\documentclass[12pt]{article}
\usepackage{geometry}
\usepackage{graphicx}
\usepackage{titlesec}
\usepackage{enumitem}
\usepackage{hyperref}
\geometry{margin=1in}
\titleformat{\section}{\normalfont\Large\bfseries}{\thesection}{1em}{}

\title{Proposal for RapidRide Ticketing System\\[0.5em] \large A Modern, Secure, and Accessible Transit Ticketing Solution}
\author{Submitted by: H. Ryott Glayzer\\Contact: harley.glayzer@mines.sdsmt.edu}
\date{\today}

\begin{document}
\maketitle
\thispagestyle{empty}
\newpage

\tableofcontents
\newpage

\section{Executive Summary}
RapidRide is a lightweight, secure, and extensible fare system for public transit. Designed for mid-sized municipalities such as Rapid City, South Dakota, it offers a cost-effective, open, and auditable alternative to commercial transit fare platforms.

Using digitally signed QR code tickets and a flexible mobile-friendly interface, RapidRide empowers riders and administrators with secure, modern infrastructure—without the costs or complexity of proprietary systems.

\section{Project Scope}
\begin{itemize}
    \item Deploy a mobile- and kiosk-friendly digital fare platform
    \item Enable digital ticket purchase via Stripe integration
    \item Support secure QR-code-based validation
    \item Provide tools for fare enforcement, reporting, and expansion
    \item Ensure accessibility for all riders, including offline and low-tech options
\end{itemize}

\section{System Overview}
\subsection{Frontend}
The RapidRide client is a cross-platform application written in Qt/QML using PySide6, providing:
\begin{itemize}
    \item A secure user wallet for ticket storage
    \item Stripe checkout integration for ticket purchasing
    \item QR code generation and display for scanned validation
    \item Offline ticket caching for intermittent connectivity
\end{itemize}

\subsection{Backend}
The backend is built in Python using FastAPI, supporting:
\begin{itemize}
    \item Ticket generation using ED25519 digital signatures
    \item Ticket validation against cryptographic and database records
    \item User login, wallet synchronization, and Stripe session handling
\end{itemize}

\section{Deployment Plan}
\begin{enumerate}
    \item \textbf{Phase 1: MVP (Proof of Concept)} \hfill 2–4 weeks
    \begin{itemize}
        \item Core ticket generation, QR validation, and Stripe checkout
        \item Basic enforcement scanner page
        \item Pilot on a small number of devices or routes
    \end{itemize}

    \item \textbf{Phase 2: Fleet Rollout and Admin Features} \hfill 4–6 weeks
    \begin{itemize}
        \item Wallet management dashboard (optional)
        \item Driver/passenger usage analytics
        \item Training and support for operators
    \end{itemize}
\end{enumerate}

\section{Cost Estimate}
\subsection*{Software Development}
\begin{itemize}
    \item Initial system deployment: \$3,500-\$6,000
    \item Optional expansion (reporting, dashboards): \$1,200–\$5,000
\end{itemize}

\subsection*{Hardware Requirements}
One MiniPC running Debian Stable is recommended for backend infrastructure. Hardware costs for such a PC range from
\$200-\$500.
The frontend client can run on existing Android tablets, iPhones, or Linux laptops with cameras for validation.


\section{Ongoing Support and Additional Development}

After deployment of the core RapidRide system, ongoing support and additional development are available at the following rates:

\subsection*{Ongoing Support}
\begin{itemize}
    \item \textbf{Standard Support (Monthly)} \hfill \$400/month
    \begin{itemize}
        \item Covers small bug fixes, basic questions, user support, and monthly maintenance updates
        \item Includes up to 4 hours/month of development time
    \end{itemize}
    \item \textbf{On-Demand Support (Hourly)} \hfill \$40/hour
    \begin{itemize}
        \item For updates, troubleshooting, or deployment help beyond the base agreement
    \end{itemize}
\end{itemize}

\subsection*{Additional Development}
New features, expansions, or tooling requests are billed at:

\begin{itemize}
    \item \textbf{\$40/hour for planned development}
    \item Flat-rate feature pricing available upon request
\end{itemize}

All code remains GPL-licensed and open to the City of Rapid City, with no proprietary restrictions.

\section{License and Code Ownership}

The RapidRide ticketing system is developed and maintained by the author and is provided under the terms of the \textbf{GNU General Public License v3 (GPL-3.0)}.

This ensures:
\begin{itemize}
    \item The source code remains free and open to the public
    \item Modifications and derivative works must also be licensed under GPL
    \item The City of Rapid City is granted full rights to use, deploy, and modify the system
    \item The developer retains ownership of the original codebase and its licensing terms
\end{itemize}

This licensing model ensures long-term transparency, prevents vendor lock-in, and aligns with the values of public digital infrastructure.

For reference, the full license text is available at: \url{https://www.gnu.org/licenses/gpl-3.0.html}



\section{Conclusion}
RapidRide provides a high-trust, low-friction fare system without vendor lock-in or heavy infrastructure costs. With a pilot-ready QR-based deployment and a path for secure digital operations, it is a strong foundation for the modernization of Rapid City public transit.

\appendix
\section{Appendix: References}
\begin{itemize}
    \item ED25519 Cryptographic Signatures: \url{https://ed25519.cr.yp.to/}
    \item Qt + PySide6 Documentation: \url{https://doc.qt.io/qtforpython/}
    \item Stripe Payments: \url{https://stripe.com/docs}
    \item FastAPI Web Framework: \url{https://fastapi.tiangolo.com}
\end{itemize}

\end{document}

