\documentclass[12pt]{article}
\usepackage{geometry}
\usepackage{graphicx}
\usepackage{titlesec}
\usepackage{enumitem}
\usepackage{hyperref}
\geometry{margin=1in}
\titleformat{\section}{\normalfont\Large\bfseries}{\thesection}{1em}{}

\title{Proposal for RapidRide Ticketing System\\[0.5em] \large A Modern, Secure, and Accessible Transit Ticketing Solution}
\author{Submitted by: H. Ryott Glayzer\\Contact: harley.glayzer@mines.sdsmt.edu}
\date{\today}

\begin{document}
\maketitle
\thispagestyle{empty}
\newpage

\tableofcontents
\newpage

\section{Executive Summary}
RapidRide is a lightweight, secure, and extensible fare system for public transit. Designed for mid-sized municipalities such as Rapid City, South Dakota, it offers a cost-effective, open, and auditable alternative to commercial transit fare platforms.

Using digitally signed QR code tickets and a flexible mobile-friendly interface, RapidRide empowers riders and administrators with secure, modern infrastructure—without the costs or complexity of proprietary systems.

\section{Project Scope}
\begin{itemize}
    \item Deploy a mobile- and kiosk-friendly digital fare platform
    \item Enable digital ticket purchase via Stripe integration
    \item Support secure QR-code-based validation
    \item Provide tools for fare enforcement, reporting, and expansion
    \item Ensure accessibility for all riders, including offline and low-tech options
\end{itemize}

\section{System Overview}
\subsection{Frontend}
The RapidRide client is a cross-platform application written in Qt/QML using PySide6, providing:
\begin{itemize}
    \item A secure user wallet for ticket storage
    \item Stripe checkout integration for ticket purchasing
    \item QR code generation and display for scanned validation
    \item Offline ticket caching for intermittent connectivity
\end{itemize}

\subsection{Backend}
The backend is built in Python using FastAPI, supporting:
\begin{itemize}
    \item Ticket generation using ED25519 digital signatures
    \item Ticket validation against cryptographic and database records
    \item User login, wallet synchronization, and Stripe session handling
\end{itemize}

\section{Deployment Plan}
\begin{enumerate}
    \item \textbf{Phase 1: MVP (Proof of Concept)} \hfill 2–4 weeks
    \begin{itemize}
        \item Core ticket generation, QR validation, and Stripe checkout
        \item Basic enforcement scanner page
        \item Pilot on a small number of devices or routes
    \end{itemize}

    \item \textbf{Phase 2: Fleet Rollout and Admin Features} \hfill 4–6 weeks
    \begin{itemize}
        \item Wallet management dashboard (optional)
        \item Driver/passenger usage analytics
        \item Training and support for operators
    \end{itemize}
\end{enumerate}

\section{Cost Estimate}
\subsection*{Software Development}
\begin{itemize}
    \item Initial system deployment: \$3,500-\$8,500
    \item Optional expansion (reporting, dashboards): \$1,200–\$5,000
\end{itemize}

\subsection*{Hardware Requirements}
One MiniPC running Debian Stable is recommended for backend infrastructure. Hardware costs for such a PC range from
\$200-\$500.
The frontend client can run on existing Android tablets, iPhones, or Linux laptops with cameras for validation.


\section{Ongoing Support and Additional Development}

After deployment of the core RapidRide system, ongoing support and additional development are available at the following rates:

\subsection*{Ongoing Support}
\begin{itemize}
    \item \textbf{Standard Support (Monthly)} \hfill \$400/month
    \begin{itemize}
        \item Covers small bug fixes, basic questions, user support, and monthly maintenance updates
        \item Includes up to 4 hours/month of development time
    \end{itemize}
    \item \textbf{On-Demand Support (Hourly)} \hfill \$40/hour
    \begin{itemize}
        \item For updates, troubleshooting, or deployment help beyond the base agreement
    \end{itemize}
\end{itemize}

\subsection*{Additional Development}
New features, expansions, or tooling requests are billed at:

\begin{itemize}
    \item \textbf{\$40/hour for planned development}
    \item Flat-rate feature pricing available upon request
\end{itemize}

All code remains GPL-licensed and open to the City of Rapid City, with no proprietary restrictions.

\section{Future Add-Ons and Expansion Opportunities}

The RapidRide system is designed to be modular and extensible. The following features can be added after the initial deployment as the needs of the transit system evolve. Each item includes a preliminary cost estimate based on hourly development and integration effort.

\subsection*{1. Admin Dashboard and Reporting}

A secure, web-based interface for transit administrators to monitor system usage, download records, and manage operational data.

\begin{itemize}
    \item \textbf{Estimated Cost:} \$1,200–\$2,500
    \item \textbf{Includes:}
    \begin{itemize}
        \item Ticket sales and usage analytics
        \item Rider activity summaries and filtering by route
        \item CSV export and dashboard visualizations
    \end{itemize}
    \item \textbf{Benefits:}
    \begin{itemize}
        \item Improves data transparency and decision-making
        \item Reduces staff time spent on manual record queries
    \end{itemize}
\end{itemize}

\subsection*{2. Offline Ticket Validation Mode}

Support for ticket validation on devices that operate without consistent internet access, such as rural-route buses or mobile fare inspectors.

\begin{itemize}
    \item \textbf{Estimated Cost:} \$800–\$1,500
    \item \textbf{Includes:}
    \begin{itemize}
        \item Local cache of ticket ID and status
        \item Secure fallback validation logic
        \item Sync mechanism for online catch-up
    \end{itemize}
    \item \textbf{Benefits:}
    \begin{itemize}
        \item Ensures continuity of fare enforcement in remote or offline areas
        \item Enables validation on the move without real-time network dependency
    \end{itemize}
\end{itemize}

\subsection*{3. Kiosk Mode for Public Purchase Stations}

A locked-down, touchscreen-compatible interface for riders to purchase tickets at transit centers or high-traffic stops.

\begin{itemize}
    \item \textbf{Estimated Cost:} \$1,000–\$1,800
    \item \textbf{Includes:}
    \begin{itemize}
        \item Touch-optimized UI layout
        \item Persistent login for kiosk identity
        \item Stripe-hosted checkout flow integration
    \end{itemize}
    \item \textbf{Requirements:}
    \begin{itemize}
        \item Commercial tablet or touch screen
        \item Optional: printer integration or SMS ticket delivery
    \end{itemize}
    \item \textbf{Benefits:}
    \begin{itemize}
        \item Expands ticket access to unbanked and non-smartphone users
        \item Reduces lines at customer service counters
    \end{itemize}
\end{itemize}

\subsection*{4. Physical QR/NFC Card Support (Tap or Scan)}

This upgrade allows riders to use physical fare cards embedded with a QR code or contactless NFC chip. Cards can be distributed to riders without smartphones, used for pass programs, or issued for rapid boarding on high-volume routes.

\begin{itemize}
    \item \textbf{Estimated Cost:} \$1,800–\$3,500
    \item \textbf{Includes:}
    \begin{itemize}
        \item Backend logic to register and validate card IDs
        \item Card issuing interface (admin or API)
        \item QR-encoded card UID integration
        \item Optional: MIFARE DESFire NFC support
    \end{itemize}
    \item \textbf{Requirements:}
    \begin{itemize}
        \item Physical card printing (typically \$1–2 per card in bulk)
        \item USB or serial NFC readers (if tap-to-ride is enabled)
        \item Fare inspector or vehicle-mounted reader device
    \end{itemize}
    \item \textbf{Benefits:}
    \begin{itemize}
        \item Enables support for riders without mobile phones
        \item Faster boarding on fixed-route buses
        \item Durable, reusable fare media for passholders or youth programs
    \end{itemize}
\end{itemize}

\subsection*{5. Multi-Language Support}

Adds language toggle functionality to the user interface, enabling the display of all content in English, Lakota, Spanish, or other languages as needed.

\begin{itemize}
    \item \textbf{Estimated Cost:} \$600–\$1,200
    \item \textbf{Includes:}
    \begin{itemize}
        \item UI translation tables
        \item Dynamic language toggle
        \item Lakota and Spanish translation integration (where available)
    \end{itemize}
    \item \textbf{Benefits:}
    \begin{itemize}
        \item Increases accessibility for non-English speakers
        \item Improves compliance with local and federal equity standards
    \end{itemize}
\end{itemize}

\subsection*{6. Inspector/Admin Tablet Mode}

Adds an inspection tool for fare enforcement personnel, with access to ticket validation tools, limited analytics, and offline fallback capability.

\begin{itemize}
    \item \textbf{Estimated Cost:} \$800–\$1,500
    \item \textbf{Includes:}
    \begin{itemize}
        \item Device-friendly layout for tablets
        \item PIN-protected admin login mode
        \item Real-time and offline ticket validation tools
    \end{itemize}
    \item \textbf{Benefits:}
    \begin{itemize}
        \item Streamlines fare enforcement on board buses
        \item Reduces dependence on paper records or verbal verification
    \end{itemize}
\end{itemize}

\subsection*{7. Ticket Sharing or Gifting}

Enables riders to transfer tickets to another user or device, either permanently (gift) or for limited use (sharing window).

\begin{itemize}
    \item \textbf{Estimated Cost:} \$700–\$1,200
    \item \textbf{Includes:}
    \begin{itemize}
        \item Ticket ownership transfer logic
        \item QR-scan or link-based acceptance flow
        \item Audit trail and one-time-use token control
    \end{itemize}
    \item \textbf{Benefits:}
    \begin{itemize}
        \item Enables gifting to friends or dependents
        \item Useful for youth programs, visitor passes, and parental sharing
    \end{itemize}
\end{itemize}

\subsection*{8. Custom Branding and Theming}

Applies visual design changes to match Rapid Transit or city branding standards.

\begin{itemize}
    \item \textbf{Estimated Cost:} \$400–\$900
    \item \textbf{Includes:}
    \begin{itemize}
        \item Custom color schemes and typefaces
        \item Logo and graphic integration
        \item Branded ticket UI and welcome screens
    \end{itemize}
    \item \textbf{Benefits:}
    \begin{itemize}
        \item Aligns visual identity with public-facing communications
        \item Creates a more professional and polished rider experience
    \end{itemize}
\end{itemize}

\vspace{1em}
\noindent All estimates are preliminary and based on a standard development rate of \$40/hour. Bundled packages, pilot programs, or grant-funded collaborations may reduce the overall cost.


\section{License and Code Ownership}

The RapidRide ticketing system is developed and maintained by the author and is provided under the terms of the \textbf{GNU General Public License v3 (GPL-3.0)}.

This ensures:
\begin{itemize}
    \item The source code remains free and open to the public
    \item Modifications and derivative works must also be licensed under GPL
    \item The City of Rapid City is granted full rights to use, deploy, and modify the system
    \item The developer retains ownership of the original codebase and its licensing terms
\end{itemize}

This licensing model ensures long-term transparency, prevents vendor lock-in, and aligns with the values of public digital infrastructure.

For reference, the full license text is available at: \url{https://www.gnu.org/licenses/gpl-3.0.html}



\section{Conclusion}
RapidRide provides a high-trust, low-friction fare system without vendor lock-in or heavy infrastructure costs. With a pilot-ready QR-based deployment and a path for secure digital operations, it is a strong foundation for the modernization of Rapid City public transit.

\appendix
\section{Appendix: References}
\begin{itemize}
    \item ED25519 Cryptographic Signatures: \url{https://ed25519.cr.yp.to/}
    \item Qt + PySide6 Documentation: \url{https://doc.qt.io/qtforpython/}
    \item Stripe Payments: \url{https://stripe.com/docs}
    \item FastAPI Web Framework: \url{https://fastapi.tiangolo.com}
\end{itemize}

\end{document}

